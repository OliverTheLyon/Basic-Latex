\documentclass{article}
\usepackage[utf8]{inputenc}

%This is the package helps with math matrices
\usepackage{amsmath}

%This is used for images
\usepackage{graphicx}

\title{Example Document}
\author{Oliver Lyon }
\date{December 2017}

\begin{document}
\maketitle

\section{Introduction}
    This is an example of how I create a Latex Document.

\section{New Section}
    We can create new section like this.
    \subsection{New Subsection}
        We can see that\\
        the slash slash cmd starts a new line. Otherwise if we write a longer sentence it will automatically wraps around and the edge of the page.
        If we start a new line int the code it will not start on a new line.\cite{ME} %this is the fast citation it's pretty awesome

        If we put one blank line in the code in between we will get a new paragraph with indentation.
    \subsection{Basic Equations}
        When it comes to Equation representations Latex is the best! We can use the begin {equation} cmd if we want a labeled equation,
        \begin{equation}
            \alpha = 2\beta + \frac{top}{bottom} 
        \end{equation}
        Or we can ask to not have the equation labeled begin {equation*},
        \begin{equation*}
            \gamma = \frac{top}{\frac{middle}{bottom}}
        \end{equation*}
        We can even do Matrices cleanly,
        \[ 
        \left( \begin{array}{cc}
            u_{1}  \\ %we brake the lines to say next element down
            \dot{u}_{1}  \\
            u_{2}  \\
            \dot{u}_{2}  \\
        \end{array} \right) 
        = 
        \left( \begin{array}{cc}
            y_1  \\
            y_2  \\
            y_3  \\
            y_4  \\
        \end{array} \right)
        \]
        \[
        \left( \begin{array}{cc} %The number of c's helps latex know how many elemnets across to give us
            u_{1} & y\\ %we brake the lines to say next element down
            \dot{u}_{1} & y\\ %this is the same notation as for tables 
            u_{2} & y\\
            \dot{u}_{2} & y\\
        \end{array} \right) 
        \]
        An import note if you try to use math notation outside of the equation Latex gets angry. You need to wrap it in Dollar signs.
        

\section{Display Code}
    To display code I find it best to use a package and an example to display it that could be found online. The other option is to use begin verbatim.
    \begin{verbatim}
Everything inside verbatim will print EXACTLY the way it is written even if it goes off the page.
    \end{verbatim}
    \begin{verbatim}
        Also best part is we can use latex cmd characters $$$
        \\\\\\\ also % make comments
    \end{verbatim}
\section{Images}
    The best guide to images is the one provided by share tex but you can control the size of the images as you put it in the image.
    
    \includegraphics[width=10cm, height=7cm]{RK3_region_of_stablility.png}
    
    images are also impacted by the paragraph blank lines. LONGER writing to demonstrate the paragraph issues
    \includegraphics[width=10cm, height=3cm]{RK3_region_of_stablility.png}

\section{Conclusion} 
    This is a very basic introduction to latex I hope this helps!
    
%it also makes bibliographies simple
\begin{thebibliography}{1}

    \bibitem{ME} {\sc Name}, {\em Title}, Credits

\end{thebibliography}

\end{document}
